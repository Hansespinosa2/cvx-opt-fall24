\section{Homework 6}

\subsection{Exercise 6.3}
Formatting the following approximation problems as LPs, QPs, SOCPs, or SDPs. $A \in \mathbb{R}^{m \times x}, b \in \mathbb{R}^m$. The rows of $A$ are denoted $a_i^\top$

\subsubsection{Part a}
Deadzone-linear penalty approximation: minimize $\sum_{i=1}^{m} \phi(a_i^\top x - b_i)$ where
\begin{equation}
    \phi(u) = \begin{cases}
        0 & |u| \leq a \\
        |u| - a & |u| > a
    \end{cases}
\end{equation}

The deadzone linear function is the same as 
\begin{equation}
    \phi(u) = \max \{ 0 , |u| - a \}
\end{equation}

The above is equivalent to

\begin{align}
  \text{minimize} & \quad t \\
  \text{subject to} & \quad t \geq 0  \\
  & \quad t \geq |u| - a
\end{align}

which is equivalent to

\begin{align}
    \text{minimize} & \quad t \\
    \text{subject to} & \quad t \geq 0  \\
    & \quad t + a \geq u \\
    & \quad t + a \geq -u
  \end{align}

Bringing it home

\begin{align}
  \text{minimize} & \quad \sum_i t_i  \\
  \text{subject to} & \quad t_i  \geq 0 & \quad \forall i  \\
  & \quad t_i  + \alpha  \geq a_i^\top x - b_i & \quad\forall i\\
  & \quad t_i + \alpha \geq - a_i^\top x + b_i & \quad\forall i 
\end{align}

We can vectorize this to be

\begin{align}
  \text{minimize} & \quad \textbf{1}^\top t \\
  \text{subject to} & \quad -t - \alpha \textbf{1} \preceq Ax - b \preceq  t + \alpha \textbf{1} \\
  & \quad t \succeq 0
\end{align}

\subsubsection{Part b}
Log-barrier penalty approximation: minimize $\sum_{i=1}^{m} \phi(a_i^\top x - b_i)$ where
\begin{equation}
    \phi(u) = \begin{cases}
        -a^2 \log (1- (\frac{u}{a})^2) & |u| < a \\
        \infty & |u| \geq a
    \end{cases}
\end{equation}

The function
\begin{equation}
    -a^2 \log (1- (\frac{u}{a})^2)
\end{equation}
is convex in $u$ as the composition order is 
\begin{gather}
    (\frac{u}{a})^2 \quad convex \\
    1- convex \quad concave \\
    \log conc(NI) \quad concave \\
    - concave \quad convex
\end{gather}

\begin{align}
  \text{minimize} & \quad \sum_{i=1}^{m} -a^2 \log(1- (\frac{a_i^\top x - b_i}{a})^2) \\
  \text{subject to} & \quad -a \leq a_i^\top x - b_i \leq a 
\end{align}

This is convex




\subsection{Exercise 6.9}
This exercise involves showing that the following problem is quasiconvex:

\begin{align}
  \text{minimize} & \quad \max_{i=1,\dots,k} |\frac{p(t_i)}{q(t_i)} - y_i|
\end{align}

where
\begin{equation}
    p(t) = a_0 + a_1 t + a_2 t^2 + \dots + a_m t^m, \quad q(t) = 1 + b_1 t + \dots + b_n t^n 
\end{equation}

and the domain of the objective function is defined as
\begin{equation}
    D = \{ (a,b) \in \mathbb{R}^{m+1} \times \mathbb{R}^n | q(t) > 0, \alpha \leq t \leq  \beta \}
\end{equation}

This problem effectively involves minimizing the maximum distance between a ratio of polynomials and measured data indexed by $i$. We are optimizing over the decision variables $a_i, b_i$. The points that we are measuring lie between $[\alpha, \beta]$ and $t_i, y_i$ are problem parameters that are given.
\\ \\
This objective function is quasiconvex if its domain is convex. The domain is convex as the only restriction on the $(a,b) \in \mathbb{R}^{m+1} \times \mathbb{R}^n$ is a lower and upper bound restriction on the given parameter $t$ and the inequality $q(t) = 1 + b_1 t + \dots + b_n t^n > 0$ this is an affine inequality over the variable and is therefore convex.
\\

\subsection{Exercise 7.1}
This exercise involves solving the maximum likelihood problem when the noise is exponentially distributed with density

\begin{equation}
    p(z) = 
    \begin{cases}
        \frac{1}{a} e^{-\frac{z}{a}} & z \geq 0 \\
        0 & z < 0
    \end{cases}
\end{equation}

The maximum likelihood problem is 
\begin{align}
  \text{maximize} & \quad \sum_{i=1}^m \log p(y_i - a_i^\top x)
\end{align}

\begin{gather}
   \log p(y_i - a_i^\top x) = \log (\frac{1}{a} e^{-\frac{y_i - a_i^\top x}{a}}) \\
   = -\log (a) -\frac{y_i - a_i^\top x}{a}
\end{gather}

\begin{gather}
    \sum_{i=1}^m -\log (a) -\frac{y_i - a_i^\top x}{a} = -m \log(a) + \frac{1}{a}\textbf{1}^\top(Ax -y) 
\end{gather}

This is maximized when we solve the LP

\begin{align}
  \text{maximize} & \quad \textbf{1}^\top (Ax-y) \\
  \text{subject to} & \quad y \succeq Ax
\end{align}

\subsection{Additional Problem HW 6.2}
This question is solved in both this pdf and the code notebook.
\\ \\
A set of $n$ teams compete in a tournament. We model each team's ability by a number $a_j \in [0,1], j = 1, \dots, n$. When teams $j$ and $k$ play each other, the probability that team $j$ wins is equal to $\textbf{prob}(a_j - a_k + v > 0)$ where $v \sim N(0, \sigma^2)$. \\
You are given the outcome of $m$ past games. These are organized as
\begin{equation}
    (j^{(i)}, k^{(i)}, y^{(i)}), \quad i = 1, \dots, m
\end{equation}
meaning that the game $i$ was played between teams $j^{(i)}, k^{(i)}$; $y^{(i)}=1$ means that team $j^{(i)}$ won, while $y^{(i)}=-1$ means that team $k^{(i)}$ won.
\subsubsection{Part a}
This part involves finding the maximum likelihood estimate of the team abilities $\hat{a} \in \mathbb{R}^n$ given the outcomes  $(j^{(i)}, k^{(i)}, y^{(i)})$.

\begin{gather}
    p(y^{(i)} | a^{(i)}) = \textbf{prob} (a_j^{(i)} - a_k^{(i)} + v > 0) \\
    = \textbf{prob} (v > a_k^{(i)} - a_j^{(i)}) =  \textbf{prob} (Z > \frac{a_k^{(i)} - a_j^{(i)}}{\sigma}) \\
    1 - \Phi (\frac{a_k^{(i)} - a_j^{(i)}}{\sigma}) = \Phi (\frac{a_j^{(i)} - a_k^{(i)}}{\sigma})
\end{gather}
Now we  want to go over all outcomes
\begin{gather}
    l(a) = \prod_{i=1}^{m} \Phi (y^{(i)}\frac{a_j^{(i)} - a_k^{(i)}}{\sigma}) \\
    l(a) = \sum_{i=1}^{m} \log \Phi (y^{(i)}\frac{a_j^{(i)} - a_k^{(i)}}{\sigma}) \\
    l(a) = \sum_{i=1}^{m} \log \Phi (\frac{1}{\sigma} (Aa)_1)
\end{gather}

The optimization problem is
\begin{align}
  \text{maximize} & \quad l(a) \\
  \text{subject to} & \quad  0 \preceq a \preceq 1
\end{align}

\subsubsection{Part b}
Solved in the notebook

\subsection{Additional Problem HW 6.3}
This question involves doing piecewise-linear fitting on one variable. Following from what the hint says, we can define parameters $\alpha, \beta$ that parametrize $y = \alpha_i x + \beta_i $ where these parameters change on each selection of data. Effectively, this problem boils down to solving a bunch of linear regressions with the added constraint of having the following linear regression begin on the same point that the previous one ended on. 
\begin{align}
  \text{minimize} & \quad \| \max_i \alpha_i x + \beta_i -y \|_2^2 \\
  \text{subject to} & \quad \alpha_{i-1} a_{i-1} + \beta_{i-1} = \alpha_{i} a_{i-1} + \beta_{i} & \forall i \in \{ 1, \dots, K \}
\end{align}

The constraint can be turned into the linear matrix equality 

\begin{align}
  C
  \begin{bmatrix}
    \alpha \\
    \beta    
  \end{bmatrix}
  = B
\end{align}

Where C is a banded $\mathbb{R}^{K \times 2 (K+1)}$matrix  with width 2 and entries $a_i, -a_i$ on each row