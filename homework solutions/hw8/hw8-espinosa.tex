\section{Homework 8}
\subsection{Exercise 9.8}
\textit{Steepest descent method in $l_\infty$-norm}. Explain how to find a steepest descent direction in the $l_\infty$-norm and give a simple interpretation.
\begin{gather}
    \Delta x_{nsd} = \arg \min \{ \nabla f(x)^\top v \quad | \quad  \| v \|_\infty \leq 1 \} \\
    \Delta x_{nsd} = - \text{sign}(\nabla f(x))
\end{gather}
\begin{equation}
    \Delta x_{sd} = \| \nabla f(x) \|_1  \Delta x_{nsd}
\end{equation}

\subsection{Exercise 10.1}
\textit{Nonsingularity of the KKT matrix}. Consider the KKT matrix
\begin{align}
  \begin{bmatrix}
     P & A^\top \\
     A & 0
  \end{bmatrix}
\end{align}
where $P \in \mathbb{S}_+^n, A \in \mathbb{R}^{p \times n}, \textbf{rank} A = p < n$
\subsubsection{Part a}
Show that each of the following statements is equivalent to nonsingularity of the KKT matrix.
\begin{itemize}
    \item $N(P) \cap N(A) = \{ 0 \}$
    \item $Ax = 0, x \neq 0 \implies x^\top P x > 0$
    \item $F^\top P F \succ 0$, where $F \in \mathbb{R} ^{n \times (n-p)}$ is a matrix for which $R(F) = N(A)$
    \item $P + A^\top Q A \succ 0$ for some $Q \succeq 0$
\end{itemize}
$N(P) \cap N(A) = \{ 0 \}$

\subsubsection{Part b}
Show that if the KKT matrix is nonsingular, then it has exactly $n$ postiive and $p$ negative eigenvalues

\subsection{Exercise 11.13} 
\textit{Self-concordance and negative entropy}.
\subsubsection{Part a}
Show that the negative entropy function $x \log x$ (on $\mathbb{R}_{++}$) is not self-concordant. \\
A function is self-concordant if 
\begin{equation}
    |f^{\prime \prime \prime}(x) | \leq 2 f^{\prime \prime} (x)^{\frac{3}{2}} \quad \forall x \in \textbf{dom} f
\end{equation}
\begin{gather}
    f^{\prime}(x) = \log x + 1 \\
    f^{\prime \prime}(x) = \frac{1}{x} \\
    f^{\prime \prime \prime}(x) = - \frac{1}{x^2} \\ 
    2 f^{\prime \prime} (x)^{\frac{3}{2}} = \frac{2}{x^{3/2}} \\
    \frac{1}{1^2} < \frac{2}{1} \\ 
    \frac{1}{\frac{1}{16}^2} = 256, \quad \frac{2}{\frac{1}{16}^{3/2}} = 128
\end{gather} 
Therefore, the function is not self-concordant.

\subsubsection{Part b}
Show that for any $t > 0$, $tx \log x - \log x $ is self-concordant (on $\mathbb{R}_{++}$)

\begin{gather}
    f^{\prime}(x) = t(\log x + 1) - \frac{1}{x} \\
    f^{\prime \prime}(x) = \frac{t}{x} + \frac{1}{x^2} \\
    f^{\prime \prime \prime}(x) = -\frac{t}{x^2} - \frac{2}{x^3} \\ 
    2 f^{\prime \prime} (x)^{\frac{3}{2}} = 2(\frac{t}{x} + \frac{1}{x^2})^\frac{3}{2} \\
    \frac{|-\frac{t}{x^2} - \frac{2}{x^3}|}{2(\frac{t}{x} + \frac{1}{x^2})^\frac{3}{2}} \leq 1 \\ 
    \frac{|\frac{tx+2}{x^3}|}{\frac{(tx+1)^\frac{3}{2}}{x^3}} \leq 2 \\
    \frac{tx+2}{(tx+1)^\frac{3}{2}} \leq 2
\end{gather} 
Here, we can see that at $ x= 0,$ the equation $h(x) = \frac{tx+2}{(tx+1)^\frac{3}{2}} = 2$. If the derivative of this function is $<0$ on all positive values of $x$, then we can state that the function is self concordant since each value would be $\geq 2$.
\begin{gather}
    h^\prime (x) = \frac{(tx+1)^\frac{3}{2} - \frac{3}{2}(tx+2) \sqrt{tx+1}}{(tx+1)^3} \\
    = - \frac{2+\frac{tx}{2}}{(tx+1)^\frac{5}{2}}
\end{gather}
Which is negative for all positive values of $t$ and $x$.